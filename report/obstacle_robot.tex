\documentclass{article}
\usepackage{amsmath}
\usepackage{graphicx}

\title{AI-Based Obstacle Avoidance Robot with Real-Time Digital Twin}
\author{Robert Anadu}
\date{}

\begin{document}
\maketitle

\section{Introduction}
This project presents a minimal autonomous mobile robot capable of real-time obstacle avoidance using sensor-based decision logic and a synchronized software simulation.

\section{System Architecture}
The system consists of a differential-drive robot executing onboard decision logic and streaming telemetry to an external visualization tool acting as a digital twin.

\section{Kinematic Model}
The robot motion follows:
\[
\dot{x} = v\cos\theta,\quad
\dot{y} = v\sin\theta,\quad
\dot{\theta} = \omega
\]

\section{Perception}
Distance measurements are obtained from an ultrasonic sensor and thresholded to determine proximity to obstacles.

\section{Decision Logic}
A finite-state controller determines behavior:
\[
\text{FORWARD if } d > 0.5,\quad
\text{TURN if } d < 0.3
\]

\section{Simulation Framework}
A Python-based simulator integrates the robot kinematics using streamed control commands, providing real-time visualization of motion and heading.

\section{Results}
The robot successfully avoids obstacles and exhibits stable autonomous behavior in both simulated and hardware-assumed modes.

\section{Limitations}
The current system assumes a single forward-facing sensor and does not account for dynamic obstacles.

\section{Future Work}
Future improvements include multi-sensor fusion, velocity profiling, and deployment on physical hardware.

\end{document}